\section{PointNet}

\begin{frame}
 \frametitle{针对无序输入的对称函数}

应用点对称函数来近似点集上的一般函数,通过多层感知器网络近似 $h$ ,通过单变量函数和最大池函数的组合近似 $g$



\begin{equation*} f(\{x_{1},\ \ldots,\ x_{n}\})\approx g(h(x_{1}),\ \ldots,\ h(x_{n}))\end{equation*}



\begin{equation*} \forall\epsilon >0,\exists h,\gamma,\quad st. \left\vert f(S)-\gamma\left(\underset{x_{i}\in S}{MAX} \{h(x_{i})\}\right)\right\vert < \epsilon \end{equation*}

    
\end{frame}

\begin{frame}
\frametitle{局部特征和全局特征的聚合}

在计算全局点云特征向量之后,通过将全局特征与每个点特征拼接起来反馈给每个点。然后我们基于拼接的点特征提取新的逐点特征,其中包含了局部信息和全局信息。
    
\end{frame}

\begin{frame}
\frametitle{仿射变换对齐网络}

通过一个T-net预测仿射变换矩阵,并直接将此变换应用于输入点云,其中T-net只由简单的全连接层组成。这一思想也可以进一步扩展到特征空间的对齐。我们可以对逐点特征应用另一个对齐网络,使用同样的方法预测特征仿射变换矩阵,以对齐来自不同输入点云的特征。为了将特征变换矩阵约束为接近正交矩阵,将正则化项添加到训练损失中。

\begin{equation*} L_{reg}=\Vert I-AA^{T}\Vert_{F}^{2},\tag{2} \end{equation*}

    
\end{frame}

\begin{frame}
\frametitle{局限性}
PointNet只能捕捉点云的全局特征,不能捕获欧氏空间点集的局部结构特征,从而限制了其识别细粒度模式的能力和对复杂场景的泛化能力。
\end{frame}

