%!TEX program = xelatex
\documentclass[aspectratio=169]{ctexbeamer}
\usepackage{multicol}
\usepackage{amsfonts,amsmath,amscd,amssymb,amsthm}
\usepackage{latexsym,bm}
\usepackage{cite}
\usepackage{mathtools,mathdots,graphicx,array,mathrsfs}
\usepackage{fancyhdr}
\usepackage{lastpage}
\usepackage{color}
\usepackage{enumitem}
\usepackage{diagbox}
\usepackage{xcolor,tcolorbox,tikz,tkz-tab,mdframed,tikz-cd}
\usepackage{framed}
\usepackage{verbatim}
\usepackage{extarrows}
\usepackage{fontspec}
\usepackage{hyperref}
\usepackage{braket}

% \usepackage{physics}  
                        %%% 宽高比说明 %%%%
%% ctexbeamer宏包支持各种宽高比,但本模板只适配了4:3(默认)和16:9的宽高比背景。
%% 添加选项aspectratio=169或aspectratio=43可以更改宽高比,默认是4:3
\usepackage[bluetheme]{ustcbeamer}
\input{ustctheme.tex}
                        %%% ustcbeamer说明 %%%%
%% 宏包使用了TikZ代码形式的背景文件(在子文件夹theme中),默认选项"bluetheme",是科大校徽的蓝色;此外ustcbeamer还内置了红色和黑色主题"redtheme","blacktheme"。

                        %%% 自定义你的主题颜色 %%%
%% 一旦使用了下述命令就会覆盖ustcbeamer的内置颜色选项,你可以设置自己喜欢的RGB色值:
% \definecolor{themecolor}{RGB}{0,150,0} % 这是绿色主题
% \definecolor{themecolor}{RGB}{0,150,150} % 青色主题,也蛮好看的

%% 注意小写rgb和大写RGB表示的色值相差255倍,即RGB{255,255,255}=rgb{1,1,1};
% \definecolor{themecolor}{rgb}{0,0.5,0.3} % 深绿色主题

%% 建议自定义的主题颜色选择偏深色
%%%%%%%%%%%%%%%%%%%%%%%%%%%%%%%%%%%%%%%%%%%%%%%%%%%%%%%%%%%%%%%%%%%%%%


\title[Algebraic structure]{
    PointNet++
}
\author[USTC]{}
\institute[USTC]{
中国科学技术大学
}
\date{\today}
\begin{document}
%\section<⟨mode specification⟩>[⟨short section name⟩]{⟨section name⟩}
%小于等于六个标题为恰当的标题

%--------------------
%标题页
%--------------------
\maketitleframe
%--------------------
%目录页
%--------------------
%beamer 101
\begin{frame}%
	\frametitle{大纲}%
   % \begin{multicols}{2} % 目录分栏
	\tableofcontents[hideallsubsections]%仅显示节
	%\tableofcontents % 显示所节和子节
    % \end{multicols}

\end{frame}%
%--------------------
%节目录页
%--------------------
\AtBeginSection[]{
\setbeamertemplate{footline}[footlineoff] % 取消页脚
  \begin{frame}%
     % \begin{multicols}{2} % 目录分栏

    \frametitle{大纲}
	%\tableofcontents[currentsection,subsectionstyle=show/hide/hide]%高亮当前节,不显示子节
    \tableofcontents[currentsection,subsectionstyle=show/show/hide]%show,shaded,hide
    % \end{multicols}

  \end{frame}
\setbeamertemplate{footline}[footlineon]%添加页脚
}
%--------------------
%子节目录页
%--------------------
\AtBeginSubsection[]{
\setbeamertemplate{footline}[footlineoff]%取消页脚
  \begin{frame}%
    \frametitle{大纲}
	%\tableofcontents[currentsection,subsectionstyle=show/hide/hide]%高亮当前节,不显示子节
    \tableofcontents[currentsection,subsectionstyle=show/shaded/hide]%show,shaded,hide
  \end{frame}
\setbeamertemplate{footline}[footlineon]%添加页脚
}


\input{doc/intro}
\section{PointNet}

\begin{frame}
 \frametitle{针对无序输入的对称函数}

应用点对称函数来近似点集上的一般函数,通过多层感知器网络近似 $h$ ,通过单变量函数和最大池函数的组合近似 $g$



\begin{equation*} f(\{x_{1},\ \ldots,\ x_{n}\})\approx g(h(x_{1}),\ \ldots,\ h(x_{n}))\end{equation*}



\begin{equation*} \forall\epsilon >0,\exists h,\gamma,\quad st. \left\vert f(S)-\gamma\left(\underset{x_{i}\in S}{MAX} \{h(x_{i})\}\right)\right\vert < \epsilon \end{equation*}

    
\end{frame}

\begin{frame}
\frametitle{局部特征和全局特征的聚合}

在计算全局点云特征向量之后,通过将全局特征与每个点特征拼接起来反馈给每个点。然后我们基于拼接的点特征提取新的逐点特征,其中包含了局部信息和全局信息。
    
\end{frame}

\begin{frame}
\frametitle{仿射变换对齐网络}

通过一个T-net预测仿射变换矩阵,并直接将此变换应用于输入点云,其中T-net只由简单的全连接层组成。这一思想也可以进一步扩展到特征空间的对齐。我们可以对逐点特征应用另一个对齐网络,使用同样的方法预测特征仿射变换矩阵,以对齐来自不同输入点云的特征。为了将特征变换矩阵约束为接近正交矩阵,将正则化项添加到训练损失中。

\begin{equation*} L_{reg}=\Vert I-AA^{T}\Vert_{F}^{2},\tag{2} \end{equation*}

    
\end{frame}

\begin{frame}
\frametitle{局限性}
PointNet只能捕捉点云的全局特征,不能捕获欧氏空间点集的局部结构特征,从而限制了其识别细粒度模式的能力和对复杂场景的泛化能力。
\end{frame}


\section{PointNet++}


\begin{frame}
\frametitle{分层点云特征学习}
    
使用分层特征学习代替PointNet中使用单个最大池化聚合点云特征。具体来说,每一个抽象层由一个采样层,一个分组层和一个学习层组成。

采样层,给定输入特征点集,使用迭代最远点采样(FPS)来选择特征点集的子集,与随机采样相比,在相同的质心数下,该方法对整个点集的覆盖率更高。

分组层,使用球查询搜索到质心点的特征距离在半径内的所有点(在实现中设置上限K),与kNN相比,球查询的局部邻域保证了固定的区域尺度,从而使局部区域特征在空间上更具泛化性,且利于学习层中的PointNet提取特征。

学习层,使用PointNet学习逐点特征,局部区域中的点的坐标首先被转换成相对于质心点的局部坐标系,通过PointNet得到该局部区域的全局特征作为下一个采样层的输入。




\end{frame}


\begin{frame}
\frametitle{分层点云特征学习}

\begin{figure}
\includegraphics[scale=0.7]{doc/img/f2.png}
\caption{2}
\end{figure}
    
\end{frame}


\begin{frame}
\frametitle{采样优化策略:非均匀采样}

多尺度分组(MSG),如图(a)所示,在每一个抽象层中使用具有不同尺度的分组层,然后根据PointNets提取每个尺度的特征。不同尺度的特征被连接以形成多尺度特征。

多分辨率分组(MRG),由于上面的MSG计算开销较大,因为它在每个质心点的大规模邻域中运行PointNet。通常由于在底层时质心点的数量相当大,因此时间成本是高昂的。MRG的层特征是两个向量的拼接。一个向量(图中左侧)是通过使用单尺度分组(SSG)汇总每个子区域的特征而获得的。另一个矢量(右)是通过使用单个PointNet直接处理局部区域中的所有原始点获得的特征。当局部区域的密度低时,左侧向量比右侧向量更不可靠,因为在计算左侧向量时子区域点云稀疏导致采样不足。在这种情况下,右侧向量的权重应该更高。相反,当局部区域的密度高时,左侧矢量提供了更精细的信息,此时应该提高左侧向量的权重。

\end{frame}

\begin{frame}
\frametitle{采样优化策略:非均匀采样}


\begin{figure}
\includegraphics[scale=0.7]{doc/img/f3.png}
\caption{3}
\end{figure}
    
    
\end{frame}


\begin{frame}
\frametitle{基于点特征传播的点云分割方法}
    
使用基于特征距离的插值传播方法逆向生成前一个抽象层所有点的近邻特征,再与前一个抽象层的分组学习特征拼接为逐点分割特征,使用PointNet的部分仿射架构进行特征学习,进行下一轮特征传播,直到传播至原始点云。在插值中,使用k个最近邻的反距离加权平均特征。


$$
f^{(j)}(x)=\frac{\sum_{i=1}^{k} w_{i}(x) f_{i}^{(j)}}{\sum_{i=1}^{k} w_{i}(x)} \quad \text { where } \quad w_{i}(x)=\frac{1}{d\left(x, x_{i}\right)^{p}}, j=1, \dots, C
$$


\end{frame}









\begin{frame}
  \frametitle{致谢}
  \centerline{\Large 谢谢!}
\end{frame}

\end{document}
